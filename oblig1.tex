\documentclass[a4paper,norsk,11pt,twoside]{article}
\usepackage[utf8]{inputenc}
\usepackage[T1]{fontenc}
\usepackage[norsk]{babel}
\usepackage{epsfig}
\usepackage{graphicx}
\usepackage{amsmath}
\usepackage{pstricks}
\usepackage{subfigure}
\usepackage{bm}
\usepackage{booktabs}       % Pakke for pene tabeller
                            % http://ctan.uib.no/macros/latex/contrib/booktabs/booktabs.pdf

\usepackage[most]{tcolorbox}

\tcbset{
    frame code={}
    center title,
    left=0pt,
    right=0pt,
    top=0pt,
    bottom=0pt,
    colback=gray!70,
    colframe=white,
    width=\dimexpr\textwidth\relax,
    enlarge left by=0mm,
    boxsep=5pt,
    arc=0pt,outer arc=0pt,
    } 

\date{11.09.2016}
\title{FYS1120 Oblig 1}
\author{Daniel Heinesen, daniehei}

\begin{document}
\maketitle
\newpage

\textbf{Oppgave 1:}

\textbf{a)}\\

Når jeg skal regne ut gradienten i oppgavene under bruker jeg at 

\begin{equation}
\nabla f =  \frac{\partial f}{\partial x}\hat{i} + \frac{\partial f}{\partial y}\hat{j} + \frac{\partial f}{\partial z}\hat{k}
\end{equation}

$i)$\\

$$
\nabla f=\nabla(x^{2}y) = 2xy\hat{i} + x^{2}\hat{j}
$$

$ii)$\\

$$
\nabla g=\nabla(xyz) = yz\hat{i} + xz\hat{j} + xy\hat{k}
$$

$iii)$\\

 

$$
\nabla h=\nabla(\frac{1}{r} e^{r^{2}}) = \frac{\partial f}{\partial r} \hat{e_r}
$$

Vi bruker produktregelen for derivasjon:

$$
= (2e^{r^{2}} - \frac{1}{r^{2}}e^{r^{2}})\hat{e_r}
$$

\textbf{b)}

Her bruker jeg at divergensen og virvlingen er
\begin{equation}
\nabla \cdot \textbf{u} = \frac{\partial}{\partial x} u_x + \frac{\partial}{\partial y} u_y + \frac{\partial}{\partial z} U_z
\end{equation}
og
\begin{equation}
\nabla\times\textbf{u}
=\begin{vmatrix}
\hat i & \hat j & \hat k \\
\frac{\partial}{\partial x} & \frac{\partial}{\partial y} & \frac{\partial}{\partial z} \\
u_x & u_y & u_z
\end{vmatrix}
\end{equation}

$i)$\\
$$
\nabla \cdot \textbf{u} = \nabla \cdot (2xy,x^{2},0) = 2x + 0 + 0 = 2x
$$

$$
\nabla\times \textbf{u} = \nabla \times(2xy,x^{2},0) = \hat{k}(2x-2x) = 0
$$

$ii)$\\

$$
\nabla \cdot \textbf{v} = \nabla \cdot (e^{yz},ln(xy),z) = 0 + x\frac{1}{xy} + 1 = 1 + \frac{1}{y}
$$

$$
\nabla\times \textbf{v} = \nabla \times(e^{yz},ln(xy),z) 
=\begin{vmatrix}
\hat i & \hat j & \hat k \\
\frac{\partial}{\partial x} & \frac{\partial}{\partial y} & \frac{\partial}{\partial z} \\
e^{yz} & ln(xy) & z
\end{vmatrix}
=\hat{j}(ye^{yz}) + \hat{k}(\frac{1}{x} - ze^{yz})
$$

$iii)$\\

$$
\nabla\cdot \textbf{w} = \nabla \cdot (yz,xz,xy) = 0 + 0 + 0 = 0
$$

$$
\nabla \times \textbf{w} = \nabla \times (yz,xz,xy) 
=\begin{vmatrix}
\hat i & \hat j & \hat k \\
\frac{\partial}{\partial x} & \frac{\partial}{\partial y} & \frac{\partial}{\partial z} \\
yz & xz & xy
\end{vmatrix}
=\left(x - x \right) \hat i + \left(y - y \right) \hat j + \left({z} - z \right) \hat k = 0
$$

$iv)$\\

$$
\nabla \cdot \textbf{a} = \nabla \cdot (y^{2}z,-z^{2}\sin(y) + 2xyz, 2z\cos(y) + y^{2}x)
$$
$$
= 0 -z^{2}\cos(y) + 2xz + 2\cos(y)
$$

$$
\nabla \times \textbf{a} = \nabla \times  (y^{2}z,-z^{2}\sin(y) + 2xyz, 2z\cos(y) + y^{2}x)
$$
$$
=\begin{vmatrix}
\hat i & \hat j & \hat k \\
\frac{\partial}{\partial x} & \frac{\partial}{\partial y} & \frac{\partial}{\partial z} \\
y^{2}z & -z^{2}\sin(y) + 2xyz & 2z\cos(y) + y^{2}x
\end{vmatrix}
$$

$$
=\left((-2zsiny + 2yx) - (-2zsiny+2xy) \right) \hat i + \left(y^2 - y^2 \right) \hat j + \left(2yz -2yz \right) \hat k 
$$

$$
\def\doubleunderline#1{\underline{\underline{#1}}}
= 0 \hat i + 0 \hat j + 0\hat{k} = 0
$$

\textbf{C)}

Matematisk sett vil $\nabla\times \textbf{v} = 0$ si at feltet er konservativt. Strømningen over en lukket kurve er null, strømning er uavhengig av vei og feltet kan skrives som $\textbf{v} = \nabla f$, hvor $f$ kalles en potens.\\

Gravitasjon er konservativt fordi arbeid er uavhengig av tid og strekning; m.a.o bare avhengig av tilbakelagt avstand(høyde). Dette betyr også at mekanisk energi er bevart.\\

I \textbf{b)} er $i)$, $iii)$ og $iv)$ konservative. $i)$ i \textbf{a)} er potensen til $i)$ i \textbf{b)}.

\textbf{d)}

Her bruker jeg at

\begin{equation}
\nabla ^{2}f = \nabla \cdot \nabla f 
\end{equation}

$i)$\\

$$
(\nabla j) = (2x + y)\hat{i} + (x + z^2)\hat{j} + 2yz\hat{k}
$$

$$
\nabla \cdot(\nabla j) = 2 + 0 + 2y = 2 + 2y
$$

$ii)$\\

Vi har fra \textbf{a)} $iii)$ at

$$
\nabla h = (2e^{r^{2}} - \frac{1}{r^{2}}e^{r^{2}})\hat{e_r}
$$

Så bruker vi at

$$
(\nabla \cdot \nabla h)_r = \frac{1}{r^{2}}\frac{\partial}{\partial r}(r^{r} \nabla h_r) = \frac{1}{r^{2}}\frac{\partial}{\partial r}(r^{r} (2e^{r^{2}} - \frac{1}{r^{2}}e^{r^{2}}))
$$

$$
= \frac{1}{r^{2}}[2r(2r^{2} - 1)e^{r^{2}} + 4re^{r^{2}}] = \frac{e^{r^{2}}}{r}(4r^{2} + 2)
$$


\textbf{Oppgave 2)}

Her skal jeg bevise for $\textbf{a} \times (\textbf{b} \times \textbf{c} ) = \textbf{b}(\textbf{a} \cdot \textbf{c} - \textbf{c}(\textbf{a} \times \textbf{b}) $ for en tilfeldig i-komponenten. Det tilsvarende vil være likt for de to andre komponentene også.

$$
[\textbf{a} \times (\textbf{b} \times \textbf{c} )]_i 
=\begin{vmatrix}
\hat i & \hat j & \hat k \\
{a_i} & {a_j} & {a_k} \\
(b_j c_k - b_k c_j) & (b_k c_i - b_i c_k) & (b_i c_j - b_j c_i)
\end{vmatrix} _i
$$

$$
= (a_j(b_i c_j - b_j c_i) - a_k(b_k c_i - b_i c_k))
$$

Så legger vi til $a_ib_ic_i - a_ib_ic_i $

$$
= (a_j(b_i c_j - b_j c_i) - a_k(b_k c_i - b_i c_k) + a_ib_ic_i - a_ib_ic_i)
$$
$$
= (b_i(a_ic_i +a_jc_j + a_kc_k) - c_i(a_ib_i + a_jb_j + a_kb_k))
$$
$$
= [\textbf{b}(\textbf{a} \cdot \textbf{c} - \textbf{c}(\textbf{a} \times \textbf{b})]_i
$$

Det tilsvarende gjøres for de andre 2 kompnentene(er helt likt, så jeg tar bare for én komponent for ikke å trenge å skrive så mye)

\textbf{Oppgave 3)}

\textbf{a)}

$$
\textbf{v} = y\hat{i} - x\hat{j} - (z-x)\hat{k}
$$

For å regne ut $\int \int_A \textbf{v} \cdot \textbf{n} dA$ enklere, regner vi det ut for hver flate av kuben og legger det sammen. For å få dette oversiktelig setter vi opp en tabell for flatene:

\begin{table}[h!]
  \centering

  \label{tab:table1}
  \begin{tabular}{c|c|c|c|c|c}
    flate & $f$ for flaten & $ \textbf{n} $ & $d\sigma$ & $\textbf{v} \times \textbf{n}$\\
    \hline
    1 & y=0 & $ -\hat j $ & dx dz & -x \\
    \hline
    2 & x=0 & $ -\hat i $ & dy dz & -y \\
    \hline
    3 & z=1 & $ \hat k $ & dx dy & 1-x \\
    \hline
    4 & y=1 & $ \hat j $ & dx dz & x \\
    \hline
    5 & x=1 & $ \hat i $ & dy dz & y \\
    \hline
    5 & z=0 & $ -\hat k $ & dx dy & x 
  \end{tabular}
\end{table}


Vi kan se med den gang at flate 1 er det motsatte av flate 4, og derfor vil bli 0 når de blir lagt sammen. Det samme gjelder flate 2 og 5. Så det vi sitter igjen med er:

$$
Q = \int \int_A \textbf{v} \cdot \textbf{n} dA =\int_0 ^{1} \int_0 ^{1}[(1-x) + x]dxdy = \int_0 ^{1} \int_0 ^{1} dxdy = 1
$$\\

\textbf{b)}

Gauss' teorem sier at 

$$
\int_A \textbf{v} \cdot \textbf{n} = \int_V \nabla \cdot \textbf{v} d\tau
$$

Her er $\nabla \cdot \textbf{v} = 1$, så

$$
Q = \int_V \nabla \cdot \textbf{v} d\tau = \int_0 ^{1} \int_0 ^{1} \int_0 ^{1} 1 dxdydz = 1
$$

Som forventet er samme svar som i \textbf{a)}


\textbf{Oppgave 4)}

$$
\textbf{w}(x,y,z) = (2x -y)\hat{i} + y^{2}\hat{j} + y^{2}z\hat{k}
$$

\textbf{a)}

$$
\nabla \cdot \textbf{w} = 2 - 2y - y^{2}
$$

\textbf{b)}

$$
\nabla \times \textbf{w} 
=\begin{vmatrix}
\hat i & \hat j & \hat k \\
\frac{\partial}{\partial x} & \frac{\partial}{\partial y} & \frac{\partial}{\partial z} \\
(2x -y) & y^{2} & y^{2}z
\end{vmatrix}
= \hat{i}(-2yz) + \hat{k}
$$


\textbf{c)}

Kurven til en sirkel med radius 1 som ligger i $z = 1$ er

$$
\bm{\Gamma } = \cos(t)\hat{i} + \sin(t)\hat{j} + 1\hat{k}
$$

Vi kan så derivere dette med hensyn på t

$$
\frac{\partial \bm{\Gamma}}{\partial t} = -\sin(t) \hat i + \cos(t) \hat j
$$

$$
\Rightarrow d\bm{\Gamma} =  (-\sin(t) \hat i + \cos(t) \hat j)dt
$$

\textbf{d)}

$$
C = \oint_{\bm{\Gamma}} \bm{W}(\bm{\Gamma}) \cdot d\bm{\Gamma}
$$

Først må vi finne $\bm{w}(\bm{\Gamma})$ og $\bm{W}(\bm{\Gamma}) \cdot d\bm{\Gamma}$

$$
\bm{w}(\bm{\Gamma}) = (2\cos(t) - \sin(t))\hat{i} - \sin^{2}(t)\hat{j} - \sin^{2}(t)\hat{k}
$$

$$
\bm{w}(\bm{\Gamma}) \cdot d\bm{\Gamma} = ((2\cos(t) - \sin(t))\hat{i} - \sin^{2}(t)\hat{j} - \sin^{2}(t)\hat{k}) \cdot (-\sin(t) \hat i + \cos(t) \hat j)
$$

$$
= 2\cos(t)\sin(t) + \sin^{2}(t) - \cos(t)\sin^{2}(t)
$$

Da får vi

$$
C = \int_0^{2\pi} (2\cos(t)\sin(t) + \sin^{2}(t) - \cos(t)\sin^{2}(t)) dt
$$

$$
= \left[\sin^{2}(t) + \frac{1}{2}(t-\sin(t)\cos(t)) - \frac{1}{3}sin^{3}(t)\right]_0^{2\pi}
$$

$$
= \pi
$$

\textbf{e)}

Stokes' teorem sier at

$$
\oint_{\bm{\Gamma}} \bm{w}(\bm{\Gamma}) \cdot d\bm{\Gamma} = \int_A \nabla \times \textbf{w} d\sigma
$$

Vi har at virvlingen er $ \nabla \times \textbf{w} = \hat{i}(-2yz) + \hat{k}$ , og $\textbf{n} = \hat{k}$(siden normalvektoren på en sirkel i xy-planet er $\hat{k}$). Da får vi

$$
C = \int_A \nabla \times \textbf{w} d\sigma = \int_0^{1} \int_0^{2\pi}(\hat{i}(-2yz) + \hat{k})\cdot \hat{k} rd\theta dr 
$$

$$
= \int_0^{1} \int_0^{2\pi} rd\theta dr = 2\pi \int_0^{1} r dr = \frac{1}{2}2\pi = \pi
$$

Som er akkurat samme svar som i \textbf{d)}.





\end{document}